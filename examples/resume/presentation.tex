%-------------------------------------------------------------------------------
%	SECTION TITLE
%-------------------------------------------------------------------------------
\cvsection{项目经历}

%-------------------------------------------------------------------------------
%	CONTENT
%-------------------------------------------------------------------------------
\begin{cventries}
	\cventry
	{} % Role
	{在线视频课程系统} % Event
	{} % Location
	{} % Date(s)
	{
		\begin{cvitems} % Description(s)
			\item {基于\textcolor{yellow}{\textbf{Spring Cloud}}+\textcolor{yellow}{\textbf{Vue}}的在线视频课程系统,采用前后端分离架构并且使用\textcolor{yellow}{\textbf{Docker Compose}}进行部署开发}
			\item {集成\textcolor{yellow}{\textbf{MyBatis Generator}}生成生成持久层代码,并且经过拓展使用\textcolor{yellow}{\textbf{FreeMarker}}模板语言做出通用的代码生成器,可以一键生成Controller、Service、Dao、前端界面以及枚举}
			\item {使用\textcolor{yellow}{\textbf{Eureka}}进行服务注册与发现,{使用\textcolor{yellow}{\textbf{Gateway}}作为网关进行对接口的统一处理}
				\item 在项目中经过不断的重构将通用的功能做成组件,例如:分页组件、文件上传组件、视频播放组件、富文本框}
			\item {在文件上传模块,完成了\textcolor{yellow}{\textbf{分片上传}}以及\textcolor{yellow}{\textbf{极速秒传}},大大节省了上传文件的时间,最后使用阿里云的OSS进行存储}
			\item {在视频播放模块,使用了阿里云的视频点播服务,对视频进行加密处理}
			\item {在通用权限设计模块没有使用权限框架,而是自己从零实现了一个权限控制模块,有利于自定义的一些拓展}
			\item {基于\textcolor{yellow}{\textbf{Redis}},实现用户的单端登录及登录校验}
		\end{cvitems}
	}
	%---------------------------------------------------------
	
	\cventry
	{} % Role
	{美食社交} % Event
	{} % Location
	{} % Date(s)
	{
		\begin{cvitems} % Description(s)
			\item {独立开发的后端基于\textcolor{yellow}{\textbf{SpringBoot}}技术栈的美食社交APP}
			\item {使用\textcolor{yellow}{\textbf{Spring Security+OAuth2}}搭建用户认证中心}
			\item {在扣库存时发生超卖现象使用\textcolor{yellow}{\textbf{Redis+Lua}}解决超卖问题}
			\item {多线程抢购时会发生一人多单问题,初始还是使用Redis+Lua解决问题后期改为使用\textcolor{yellow}{\textbf{Radisson}}作为分布式锁解决一人多单问题}
			\item {用户关注会推送消息,类似朋友圈,根据时间排序使用\textcolor{yellow}{\textbf{Sorted Set}}实现Feed流功能}
			\item {用户签到如果使用传统数据库MySQL占用的资源太多使用高阶数据类型\textcolor{yellow}{\textbf{Bitmap}}完成用户签到功能}
		\end{cvitems}
	}
	\cventry
	{} % Role
	{wiki知识库系统} % Event
	{} % Location
	{} % Date(s)
	{
		\begin{cvitems} % Description(s)
			\item {独立开发的\textcolor{yellow}{\textbf{SpringBoot}}+\textcolor{yellow}{\textbf{Vue}}的知识库系统,采用前后端分离架构}
			\item {集成\textcolor{yellow}{\textbf{MyBatis}}及其官方代码生成器\textcolor{yellow}{\textbf{MyBatis Generator}}生成持久层代码,MyBatis分页插件\textcolor{yellow}{\textbf{PageHelper}}}
			\item {集成支持Vue3的UI组件库\textcolor{yellow}{\textbf{Ant Design Vue}},完成网站页面基本布局,提高了开发效率,减少了\textcolor{yellow}{\textbf{50\%}}的开发时间}
			\item {使用IDEA的HTTP Client进行RESTful API进行接口测试,通过后端校验框架\textcolor{yellow}{\textbf{Validation}}进行参数校验}
			\item {集成轻量级富文本框\textcolor{yellow}{\textbf{wangEditor}},支持在知识库中插入图片、视频等}
			\item {对接分布式缓存\textcolor{yellow}{\textbf{Redis}},实现用户登录及登录校验}
			\item {使用SpringBoot定时任务进行定时统计,集成\textcolor{yellow}{\textbf{WebSocket}}完成网站通知,增加数据统计并集成报表组件\textcolor{yellow}{\textbf{ECharts}}完成相关报表展示,丰富首页内容,最后通过\textcolor{yellow}{\textbf{Nginx}}发布项目}
		\end{cvitems}
	}
	%---------------------------------------------------------

	%---------------------------------------------------------
	\cventry
	%---------------------------------------------------------
	{该项目包含了门户平台、媒体中心、运营中心三大平台,主要包含用户模块、文件模块、文章模块、管理模块} % Role
	{自媒体平台} % Event
	{} % Location
	{} % Date(s)
	{
		\begin{cvitems} % Description(s)
			\item {独立开发的\textcolor{yellow}{\textbf{SpringBoot}}+\textcolor{yellow}{\textbf{Vue2}}的具有门户平台+媒体中心+运营中心的自媒体平台,采用前后端分离架构}
			\item {集成腾讯云的短信服务完成一键登录注册功能,阿里云的\textcolor{yellow}{\textbf{OSS}}}进行用户头像保存,门户中心使用人脸识别进行登录以及AI文本检测对内容安全审核
			\item {集成HTTP组件\textcolor{yellow}{\textbf{Axios}}与后端进行交互}
			\item {通过后端校验框架\textcolor{yellow}{\textbf{Validation}}进行不同场景的参数校验}
			\item {集成富文本框\textcolor{yellow}{\textbf{Summernote}},进行文章、图片以及视频的上传}
			\item {对接分布式缓存\textcolor{yellow}{\textbf{Redis}},实现用户登录,登录校验,缓存用户信息,阅读数,点赞数,粉丝关注数等}
			\item {使用SpringBoot定时任务进行文章定时发布,集成\textcolor{yellow}{\textbf{RabbitMQ}}进行解耦避免定时任务的全表扫描,\textcolor{yellow}{\textbf{ECharts}}完成展示男女粉丝比例以及地域信息}	
			\item {集成\textcolor{yellow}{\textbf{MongoDB}}来保存友情链接,\textcolor{yellow}{\textbf{GridFS}}保存用户人脸信息}
			\item {集成\textcolor{yellow}{\textbf{Freemarker}}详情页数据静态化,将文件上传到\textcolor{yellow}{\textbf{GridFS}}}
			\item
			{集成\textcolor{yellow}{\textbf{SpringCloud}}组件,\textcolor{yellow}{\textbf{Ribbon+Feign}}负载均衡,\textcolor{yellow}{\textbf{Zuul}}配置网关,\textcolor{yellow}{\textbf{Config+Bus}}统一配置中心}
		\end{cvitems}
	} 
\end{cventries}
