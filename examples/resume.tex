%!TEX TS-program = xelatex
%!TEX encoding = UTF-8 Unicode
% Awesome CV LaTeX Template for CV/Resume
%
% This template has been downloaded from:
% https://github.com/posquit0/Awesome-CV
%
% Author:
% Claud D. Park <posquit0.bj@gmail.com>
% http://www.posquit0.com
%
% Template license:
% CC BY-SA 4.0 (https://creativecommons.org/licenses/by-sa/4.0/)
%


%-------------------------------------------------------------------------------
% CONFIGURATIONS
%-------------------------------------------------------------------------------
% A4 paper size by default, use 'letterpaper' for US letter
\documentclass[11pt, letterpaper]{awesome-cv}
\usepackage[UTF8]{ctex}
%\usepackage[colorlinks,linkcolor=red]{hyperref}
\usepackage{soul}
\usepackage{color}
\definecolor{yellow}{RGB}{255,102,0}
% Configure page margins with geometry
\geometry{left=1.4cm, top=.8cm, right=1.4cm, bottom=1.8cm, footskip=.5cm}

% Specify the location of the included fonts
\fontdir[fonts/]
% Color for highlights
% Awesome Colors: awesome-emerald, awesome-skyblue, awesome-red, awesome-pink, awesome-orange
%                 awesome-nephritis, awesome-concrete, awesome-darknight

\colorlet{awesome}{awesome-orange}

% Uncomment if you would like to specify your own color
% \definecolor{awesome}{HTML}{CA63A8}

% Colors for text
% Uncomment if you would like to specify your own color
% \definecolor{darktext}{HTML}{414141}
% \definecolor{text}{HTML}{333333}
% \definecolor{graytext}{HTML}{5D5D5D}
% \definecolor{lighttext}{HTML}{999999}

% Set false if you don't want to highlight section with awesome color
% 这里是设置要不要高亮显示,不想显示就写false,想显示就true,上面配置哪个主题的高亮
\setbool{acvSectionColorHighlight}{false}
% If you would like to change the social information separator from a pipe (|) to something else
\renewcommand{\acvHeaderSocialSep}{\quad\textbar\quad}
%工作经历,职位那边写的要不就纯大写了
\renewcommand{\entrypositionstyle}[1]{{\fontsize{8pt}{1em}\bodyfont\color{graytext} #1}}
%设置排版间距大小
%\renewcommand{\normalsize}{\fontsize{12}{12}\selectfont}
%\newcommand*{\letteropeningstyle}[1]{{\fontsize{15pt}{1em}\bodyfontlight\bfseries\color{black} #1}}
%-------------------------------------------------------------------------------
%	PERSONAL INFORMATION
%	Comment any of the lines below if they are not required
%-------------------------------------------------------------------------------
% Available options: circle|rectangle,edge/noedge,left/right
%\photo[rectangle,edge,right]{./examples/profile}
\name{}{侯江涛}
%这个就是要不要显示你的职位,橙色的
\position{{\enskip\cdotp\enskip}}
%\address{42-8, Bangbae-ro 15-gil, Seocho-gu, Seoul, 00681, Rep. of KOREA}

\mobile{13027595769}
\github{saveunhappy}
\email{houjiangtaoay@foxmail.com}
%\homepage{www.posquit0.com}

%\linkedin{posquit0}
% \gitlab{gitlab-id}
% \stackoverflow{SO-id}{SO-name}
% \twitter{@twit}
% \skype{skype-id}
% \reddit{reddit-id}
% \medium{madium-id}
% \googlescholar{googlescholar-id}{name-to-display}
%% \firstname and \lastname will be used
% \googlescholar{googlescholar-id}{}
% \extrainfo{extra informations}

%\quote{``Be the change that you want to see in the world."}


%-------------------------------------------------------------------------------
\begin{document}

% Print the header with above personal informations
% Give optional argument to change alignment(C: center, L: left, R: right)
\makecvheader[C]

% Print the footer with 3 arguments(<left>, <center>, <right>)
% Leave any of these blank if they are not needed
%\makecvfooter
%  {\today}
%  {Byungjin Park~~~·~~~Résumé}
%  {\thepage}


%-------------------------------------------------------------------------------
%	CV/RESUME CONTENT
%	Each section is imported separately, open each file in turn to modify content
%-------------------------------------------------------------------------------
%-------------------------------------------------------------------------------
%	SECTION TITLE
%-------------------------------------------------------------------------------
\cvsection{个人总结}


%-------------------------------------------------------------------------------
%	CONTENT
%-------------------------------------------------------------------------------
\begin{cvparagraph}

%---------------------------------------------------------
热爱钻研,擅长阅读源码来解决问题,近一年在\textcolor{yellow}{\textbf{GitHub}}有\textcolor{yellow}{\textbf{1400}}次提交,\textcolor{yellow}{\textbf{给世界顶级项目Apache Maven、SpringBoot、Gradle、alibaba/fastjson贡献过代码}}。通过阅读《重构:改善既有代码的设计》和《Effective Java》使自己的技术水平又有了进一步的提升,对底层原理比较好奇,出于兴趣,实现了一个编译器的部分功能。

\end{cvparagraph}
%---------------------------------------------------------



%%-------------------------------------------------------------------------------
%	SECTION TITLE
%-------------------------------------------------------------------------------
\cvsection{Skills}


%-------------------------------------------------------------------------------
%	CONTENT
%-------------------------------------------------------------------------------
\begin{cvskills}

%---------------------------------------------------------
  \cvskill
    {后端} % Category
    {Spring, SpringMVC, MyBatis, SpringBoot ,Spring Cloud主流技术栈} % Skills

%---------------------------------------------------------
  \cvskill
    {数据库} % Category
	{MySQL(MariaDB), Redis} % Skills

%---------------------------------------------------------
  \cvskill
    {项目构建工具} % Category
    {Maven, Gradle} % Skills

%---------------------------------------------------------
  \cvskill
    {版本控制工具} % Category
    {Git} % Skills
%---------------------------------------------------------
   \cvskill
    {运维} % Category
    {Linux, Docker } % Skills
%---------------------------------------------------------
    \cvskill
    {其他} % Category
    {熟悉常用的数据结构及网络协议 } % Skills
	

%---------------------------------------------------------
\end{cvskills}

%-------------------------------------------------------------------------------
%	SECTION TITLE
%-------------------------------------------------------------------------------
\setstcolor{green}
\cvsection{个人技能}


%-------------------------------------------------------------------------------
%	CONTENT
%-------------------------------------------------------------------------------
\begin{cventries}

%---------------------------------------------------------
  \cventry
    {} % Job title
    {} % Organization
    {} % Location
    {} % Date(s)
    {
      \begin{cvitems} % Description(s) of tasks/responsibilities
      	\item {熟练掌握Java语言以及面向对象设计思想,具有扎实的Java编程功底和编码规范}
      	\item {熟悉Spring、SpringMVC、MyBatis、Spring Boot等开源框架做整合开发}
        \item {熟悉Spring Boot+Spring Cloud的微服务架构,了解Spring的IOC、AOP的编程思想}
        \item {熟悉MySQL(MariaDB)关系型数据库,理解主从复制读写分离的原理,对MongoDB有一定的了解}
        \item {熟悉Redis非关系型数据库,了解RDB和AOF方式}
        \item {熟练使用Maven、Gradle进行项目依赖管理}
        \item {熟悉Linux、Docker等常用命令可以独立部署项目}
        \item {熟悉常用数据结构如:数组、链表、栈、队列、二叉树、哈希表、堆}
      \end{cvitems}
    }

%---------------------------------------------------------
\end{cventries}

%-------------------------------------------------------------------------------
%	SECTION TITLE
%-------------------------------------------------------------------------------
\setstcolor{green}
\cvsection{工作经历}


%-------------------------------------------------------------------------------
%	CONTENT
%-------------------------------------------------------------------------------
\begin{cventries}

%---------------------------------------------------------
  \cventry
    {Java开发工程师} % Job title
    {上海亦贴网络科技有限公司} % Organization
    {上海} % Location
    {2019.7 - 2020.6} % Date(s)
    {
      \begin{cvitems} % Description(s) of tasks/responsibilities
      	\item {负责平台程序维护和新增功能二次开发}
     	\item {协助完成日常服务运维、部署等工作,保证服务的稳定运行}
    %  	\item {在应用开发部门从事管理系统开发,主要参与故障处理平台建设}
    %  	\item {完成故障处理、版本升级功能的需求分析, 数据库设计以及后端实现}
      \end{cvitems}
    }
  \cventry
	{Java开发工程师} % Job title
	{上海金丘信息科技有限公司} % Organization
	{上海} % Location
	{2022.1 - 2022.6} % Date(s)
{
	\begin{cvitems} % Description(s) of tasks/responsibilities
	%	\item {在上级的领导和监督下完成量化的工作要求}
		\item {根据开发进度和任务分配完成相应模块软件的设计开发编程任务}
		\item {进行程序单元、功能的测试查出软件存在的缺陷并保证其质量}
	%	\item {维护软件使之保持可用性和稳定性}
	\end{cvitems}
}


%---------------------------------------------------------
\end{cventries}

%\input{resume/honors.tex}
%%-------------------------------------------------------------------------------
%	SECTION TITLE
%-------------------------------------------------------------------------------
\cvsection{个人技能}

%-------------------------------------------------------------------------------
%	CONTENT
%-------------------------------------------------------------------------------
\begin{cventries}
	
	%---------------------------------------------------------
	\cventry
	{} % Role
	{  
	} % Event
	{} % Location
	{} % Date(s)
	{
		\begin{cvitems} % Description(s)
			 {感谢您花时间阅读我的简历,期待能有机会和您共事。}
		\end{cvitems}
	}

	
\end{cventries}

%-------------------------------------------------------------------------------
%	SECTION TITLE
%-------------------------------------------------------------------------------
\cvsection{项目经历}

%-------------------------------------------------------------------------------
%	CONTENT
%-------------------------------------------------------------------------------
\begin{cventries}
	
	%---------------------------------------------------------
	\cventry
	{用户入口(点击查看): \url{http://o2o.hjtwebsite.top/o2o/frontend/index} \item 
		店家入口(点击查看): \url{http://o2o.hjtwebsite.top/o2o/local/login}} % Role
	{校园商铺O2O} % Event
	{} % Location
	{} % Date(s)
	{
		\begin{cvitems} % Description(s)
			\item {独立开发的基于\textcolor{yellow}{\textbf{SSM}}技术栈的校园商铺平台,前端使用\textcolor{yellow}{\textbf{SUI Mobile}}UI库进行快速搭建}
			\item {为加快迭代速度,后端由SSM迁移至\textcolor{yellow}{\textbf{SpringBoot}}}
			\item {为了提升业务系统性能,优化用户体验,\textcolor{yellow}{\textbf{MySQL}}进行主从同步配置,项目读写分离,引入\textcolor{yellow}{\textbf{Redis}}缓存加快了请求响应速度}
			\item {使用图片开源工具\textcolor{yellow}{\textbf{Thumbnailator}}进行图片处理,验证码组件\textcolor{yellow}{\textbf{Kaptcha}}完成注册、登录功能,作业调度框架\textcolor{yellow}{\textbf{Quartz}}进行每日销量统计}
		\end{cvitems}
	}
	%---------------------------------------------------------
	\cventry
	{项目地址(点击查看):\url{http://wiki.hjtwebsite.top/}} % Role
	{wiki知识库系统} % Event
	{} % Location
	{} % Date(s)
	{
		\begin{cvitems} % Description(s)
			\item {独立开发的\textcolor{yellow}{\textbf{SpringBoot}}+\textcolor{yellow}{\textbf{Vue3}}的知识库系统,采用前后端分离架构}
			\item {集成\textcolor{yellow}{\textbf{MyBatis}}及其官方代码生成器\textcolor{yellow}{\textbf{MyBatis Generator}}生成持久层代码,MyBatis分页插件\textcolor{yellow}{\textbf{PageHelper}}}
			\item {集成支持Vue3的UI组件库\textcolor{yellow}{\textbf{Ant Design Vue}},完成网站页面基本布局,提高了开发效率,减少了\textcolor{yellow}{\textbf{50\%}}的开发时间}
			\item {集成HTTP组件\textcolor{yellow}{\textbf{Axios}},解决前后端分离架构中常见的问题,如跨域、参数传递、多环境配置等}
			\item {使用IDEA的HTTP Client进行RESTful API进行接口测试,通过后端校验框架\textcolor{yellow}{\textbf{Validation}}进行参数校验}
			\item {集成轻量级富文本框\textcolor{yellow}{\textbf{wangEditor}},支持在知识库中插入图片、视频等}
			\item {对接分布式缓存\textcolor{yellow}{\textbf{Redis}},实现用户登录及登录校验}
			\item {使用SpringBoot定时任务进行定时统计,集成\textcolor{yellow}{\textbf{WebSocket}}完成网站通知,增加数据统计并集成报表组件\textcolor{yellow}{\textbf{ECharts}}完成相关报表展示,丰富首页内容,最后通过\textcolor{yellow}{\textbf{Nginx}}发布项目}
		\end{cvitems}
	}
	%------------------------------------------------------
	\cventry
	{负责后端接口的编写} % Role
	{智慧督察系统} % Event
	{} % Location
	{} % Date(s)
	{
		\begin{cvitems} % Description(s)
			\item {基于\textcolor{yellow}{\textbf{SpringBoot}}+\textcolor{yellow}{\textbf{React}}的智慧督察系统,采用前后端分离架构并且使用\textcolor{yellow}{\textbf{Docker Compose}}进行部署开发}
			\item {集成\textcolor{yellow}{\textbf{MyBatis-Plus}}进行高效敏捷开发持久层代码}
			\item {使用\textcolor{yellow}{\textbf{Swagger UI}}实现接口文档自动生成}
			\item {\textcolor{yellow}{\textbf{Shiro}}和\textcolor{yellow}{\textbf{JWT}}进行用户校验和授权}
			\item {使用\textcolor{yellow}{\textbf{MinIO}}进行对象存储}
		\end{cvitems}
	}
	\cventry
	{负责后端接口的编写} % Role
	{自媒体平台} % Event
	{} % Location
	{} % Date(s)
	{
		\begin{cvitems} % Description(s)
			\item {独立开发的\textcolor{yellow}{\textbf{SpringBoot}}+\textcolor{yellow}{\textbf{Vue2}}的具有门户平台+媒体中心+运营中心的自媒体平台,采用前后端分离架构}
			\item {集成\textcolor{yellow}{\textbf{MyBatis}}和通用\textcolor{yellow}{\textbf{Mapper}}生成持久层代码,MyBatis分页插件\textcolor{yellow}{\textbf{PageHelper}}}
			\item {集成腾讯云的短信服务完成一键登录注册功能,阿里云的\textcolor{yellow}{\textbf{OSS}}}进行用户头像保存,门户中心使用人脸识别进行登录以及AI文本检测对内容安全审核
			\item {集成HTTP组件\textcolor{yellow}{\textbf{Axios}}与后端进行交互}
			\item {通过后端校验框架\textcolor{yellow}{\textbf{Validation}}进行不同场景的参数校验}
			\item {集成富文本框\textcolor{yellow}{\textbf{Summernote}},进行文章、图片以及视频的上传}
			\item {对接分布式缓存\textcolor{yellow}{\textbf{Redis}},实现用户登录,登录校验,缓存用户信息,阅读数,点赞数,粉丝关注数等}
			\item {使用SpringBoot定时任务进行文章定时发布,集成\textcolor{yellow}{\textbf{RabbitMQ}}进行解耦避免定时任务的全表扫描,\textcolor{yellow}{\textbf{ECharts}}完成展示男女粉丝比例以及地域信息}	
			\item {集成\textcolor{yellow}{\textbf{MongoDB}}来保存友情链接,\textcolor{yellow}{\textbf{GridFS}}保存用户人脸信息}
			\item {集成\textcolor{yellow}{\textbf{Freemarker}}详情页数据静态化,将文件上传到\textcolor{yellow}{\textbf{GridFS}}}
			\item
			{集成\textcolor{yellow}{\textbf{SpringCloud}}组件,\textcolor{yellow}{\textbf{Ribbon+Feign}}负载均衡,\textcolor{yellow}{\textbf{Zuul}}配置网关,\textcolor{yellow}{\textbf{Config+Bus}}统一配置中心}
		\end{cvitems}
	}
\end{cventries}


%\input{resume/writing.tex}
%\input{resume/committees.tex}
%-------------------------------------------------------------------------------
%	SECTION TITLE
%-------------------------------------------------------------------------------
\cvsection{其他链接}


%-------------------------------------------------------------------------------
%	CONTENT
%-------------------------------------------------------------------------------
\begin{cventries}
	
	%---------------------------------------------------------
	\cventry
	{} % Degree
	{} % Institution
	{} % Location
	{} % Date(s)
	{
		\begin{cvitems} % Description(s) bullet points
		 \item {我的博客: \url{https://www.yuque.com/u646146-axqht/iz4znd}}
		 \item {我的GitHub: \url{https://github.com/saveunhappy}} % Role
		\end{cvitems}
	}
	
	%---------------------------------------------------------
\end{cventries}

%-------------------------------------------------------------------------------
%	SECTION TITLE
%-------------------------------------------------------------------------------
\cvsection{Education}


%-------------------------------------------------------------------------------
%	CONTENT
%-------------------------------------------------------------------------------
\begin{cventries}

%---------------------------------------------------------
  \cventry
    {B.S. in Computer Science and Engineering} % Degree
    {POSTECH(Pohang University of Science and Technology)} % Institution
    {Pohang, S.Korea} % Location
    {Mar. 2010 - Aug. 2017} % Date(s)
    {
      \begin{cvitems} % Description(s) bullet points
        \item {Got a Chun Shin-Il Scholarship which is given to promising students in CSE Dept.}
      \end{cvitems}
    }
\hl{This will be highlight.}

%---------------------------------------------------------
\end{cventries}


%%-------------------------------------------------------------------------------
%	SECTION TITLE
%-------------------------------------------------------------------------------
\cvsection{个人技能}

%-------------------------------------------------------------------------------
%	CONTENT
%-------------------------------------------------------------------------------
\begin{cventries}
	
	%---------------------------------------------------------
	\cventry
	{} % Role
	{  
	} % Event
	{} % Location
	{} % Date(s)
	{
		\begin{cvitems} % Description(s)
			 {感谢您花时间阅读我的简历,期待能有机会和您共事。}
		\end{cvitems}
	}

	
\end{cventries}

%\input{resume/extracurricular.tex}


%-------------------------------------------------------------------------------
\end{document}
